\section{Introduction}


Due to the popularity of 3D scanners, meshes are becoming more and more accessible.
However, the influence of environment and people makes the data contain a lot of high frequency noises.
Mesh filtering is a vital preprocessing tool of updating vertex positions in a mesh to achieve perfect goals like denoising, smoothing or enhancement.
It typically preserve the obvious geometric structures, while undesirable noise need be discarded.
Although a variety of mesh filtering methods achieve satisfactory results \cite{jones2003non, zheng2011bilateral, he2013mesh, Zhang2015Filter, Wang2015rolling},
they have a little imperfect results in dealing with the geometry features such as edges and corners.


The bilateral filter, introduced in \cite{tomasi1998bilateral}, is a famous edge-preserving image filter, which prefers near values to distant values in both domain and range.
Unlike bilateral filter, the guided filters \cite{Petschnigg2004, he2010guided} set the filtering weights using the intensity difference from another image and has better edge-preserving property.
Due to their success in image processing and computational photography, 
researchers make many attempts use them to geometry processing such as mesh denoising and smoothing \cite{jones2003non, zheng2011bilateral, Solomon2014general, Zhang2015Filter}. 
The attempt, considering normals as a surface signal defined over the original mesh, makes a great process in mesh denoising.
The main reason is that the normal differences are very bigger when crossing the feature edges.
However, they still have the disadvantage in dealing with the narrow edges, which are important and widespread feature in meshes.
For example, choosing a large neighborhood will result in smoothing the edge, while selecting a small one would limit the ability in removing denoise.
Their common defect is that computing the filtering weight only considers the signal differences between the current point and one of its neighbours, ignoring the relation of the two points.
In our intrinsic filtering model, we build the connections through a geodesic path to effectively solve this problem.


Although the geodesic filter~\cite{grazzini2009edge} and propagated image filter \cite{Chang2015propagated} discovers the shortcoming, 
estimate the filtering weight accumulated differences along the values on the geodesic path.
Their difference is \cite{Chang2015propagated} considers two kinds of accumulative difference, another only thinks of one kind. 
As the accumulative difference has the ability to adapt the image context, these two filters both have a superior edge-preserving property.
Nevertheless, their work is only on the image denoising.
They don't provide a generalized model for solving similar problems.
In our paper, we introduce a universal filtering method for the geometry signal (position, normal, curvature and so on) on 2d manifold surface.
Afterwards, we apply our model to mesh denoising and verify the effectiveness of our method.
Furthermore, based on the mesh's own properties, the operation of determining the filtering weight is more reasonable than in images. 
In addition, \cite{grazzini2009edge} and \cite{Chang2015propagated} are both a special case of our model.


In this paper, based on the theory of signal filter, we propose a generalized intrinsic filtering model for 2D manifold. 
For applying this model to mesh denoising, 
similar to the previous method, we consider face normals as a surface signal defined over a original mesh.
Then we calculates the filtering weight through using accumulated normal difference along the geodesic path connecting face barycenters.
Mesh features are better retained. 
In addition, we introduce a particular pattern for solving the time deficiency in applying geodesic algorithm and obtain even better experimental results.
The effectiveness of our approach is illustrated by extensive experimental results in mesh denoising.

