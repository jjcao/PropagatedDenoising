\section{Introduction}


Due to the popularity of 3D scanners, meshes are becoming more and more accessible.
However, the influence of environment and people makes the data contain a lot of high frequency noises.
Mesh filtering is a vital preprocessing tool of updating vertex positions in a mesh to achieve perfect goals like denoising, smoothing or enhancement.
It typically preserve the obvious geometric structures, while undesirable noise need be discarded.
Although a variety of mesh filtering methods achieve satisfactory results \cite{fleishman2003bilateral, zheng2011bilateral},
they have a little imperfect results in dealing with the geometry features such as edges and corners.
Because the characteristics of adjacent regions are blended, the output mesh would be imperfect.


The bilateral filter, introduced in \cite{tomasi1998bilateral}, is a famous edge-preserving image filter, which prefers near values to distant values in both domain and range.
Unlike bilateral filter, the guided filters \cite{Petschnigg2004, he2010guided} set the filtering weights using the intensity difference from another image and has better edge-preserving property.
Due to their success in image processing and computational photography and considering normals as a surface signal defined over the original mesh,
many attempts have been made to adapt them to geometry processing such as mesh denoising and smoothing \cite{jones2003non, zheng2011bilateral, Solomon2014general}.
These filter are able to alleviate the smooth problem, but they still have disadvantages in the process of preserving image/mesh context.
For example, choosing a large neighborhood will result in blend of cross-region, while selecting a small one would limit the filtering ability.%performance.
In our intrinsic filtering framework, we build the connections between filtering signal and its neighbors to effectively solve this problem.

The geodesic filter~\cite{grazzini2009edge}, unlike the bilateral filter directly uses the signal differences, considers the accumulated difference between signals on a geodesic path.
However, it only uses the adjacent accumulated differences.
Our intrinsic filter simultaneously apply another integral differences between signals.
Recently,  propagated image filter \cite{Chang2015propagated} was addressed and display its powerful filtering performance.
It calculates the accumulated difference not only between the values of adjacent pixels, but also start-end pixels(??) in choosing shortest path.
Then it dynamically determines their filtering weight by these two accumulated difference.
As the accumulative difference has the ability to estimate the image context, the propagated filter has a more superior edge-preserving property.
From our paper, we find that propagated filter is only a special case of our method.

In this paper, inspired by bilateral filter, then based on the theory of signal filtering, we propose a novel filtering method called intrinsic filter for 2D manifold surface.
Similar to the previous method, we consider some geometry feature (normals, Gaussian curvature, position and so on) as a surface signal defined over a 2D manifold surface.
We calculates the filtering weights through using curve integral along the geodesic path.
Then the weights are applied to traditional mesh denoising framework and achieve wonderful results.
In addition, we introduce a particular pattern for solving the time defeciency in applying geodesic algorithm and obtain even better experimental results.
The effectiveness of our approach is illustrated by extensive experimental results in mesh denoising.

