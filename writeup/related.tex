\section{Relate work}

The rise of 3D scanning devices makes captured meshes become more and more easier.
However, as the influence of light or devices, meshes which are taken often contain high-frequency noises.
Thus mesh denoising, as an important tool of geometry processing, becomes a popular study point~\cite{Wang2008comprehensive}.
Due to the large amount of research work in this field, we introduce some methods that have similarities with ours.

The purpose of mesh denoising is removing the noise without damaging the true structure.
Because the edge-preserving property of bilateral filter~\cite{tomasi1998bilateral}, it is widely used in image processing~\cite{oh2001image, durand2002fast, Barash2004common, Wang2014decoupling} and geometry processing such as mesh denoising~\cite{fleishman2003bilateral, jones2003non, zheng2011bilateral, Solomon2014general} and mesh feature enhancement~\cite{Wang2006bilateral}.

The papers \cite{fleishman2003bilateral} and ~\cite{jones2003non} apply bilateral filter to the mesh vertex positions as it is used in image denoising~\cite{durand2002fast}.
\cite{zheng2011bilateral} employ the bilateral filter to the mesh face normals, then according the filtered normals the process of vertex updated is implemented.
\cite{Solomon2014general} also perform mesh denoising by filtering the face normals, using a generalized cross-bilateral filter.
Our denoising algorithm also applies the similar approach which iterates face normal filtering and vertex updates.
However, our method differs from \cite{zheng2011bilateral} and \cite{Solomon2014general} which both only consider the spatial and range distances that may smooth the weak edges,
also is different from~\cite{Zhang2015Filter} which uses a joint bilateral filter on the normals that may break the sharp corners.
As we build the connections which reflected in the geodesic path between face normals, obtaining the more accurate weight for filtering algorithm.

The two-stage process including filtering face normals and updating vertex positions has also been adapted by many famous work~\cite{yagou2002mesh, chen2005sharpness, sun2008random, Wang2015rolling}.
The mainly differences between these methods is in their normal filtering strategies.
Mean and median filtering and rolling guidance filter~\cite{Zhang2014rolling} are applied in~\cite{yagou2002mesh} and \cite{Wang2015rolling} respectively.
\cite{chen2005sharpness} automatically select filters according the mesh local sharpness.
In \cite{sun2008random}, face normal filtering is performed by weighted averaging of normals based on the concept of random walks.

Another type of denoising methods employ different strategies to filter the type of vertices which belonging to corner, edge or flat areas on a mesh.
The paper \cite{bian2011feature} classify the vertices based on the volume integral invariant
while \cite{fan2010robust} uses a local quadric model to fit the vertices and curvature tensor for obtaining the types of vertices.
These two papers \cite{wang2012cascaded} and \cite{wei2015bi} apply different normal estimation strategies to remove noise.

Other researchers \cite{he2013mesh} apply sparsity optimization to mesh denoising and get a success because the sharp structures are usually sparse on the mesh.
They apply $L_0$ minimization to an edge-based Laplacian operator and effectively preserve the sharp structure of mesh.
However, the optimization algorithm prefer flat shapes and may not be suitable for complex meshes.
The paper \cite{Wang2014decoupling} uses $L_1$ optimization to recover sharp structures from noisy meshes and also has above-mentioned problems.






