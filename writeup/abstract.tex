\begin{abstract}

%Most mesh denoising methods are weighted average normal filters and their performance depends on the weight design.
%When computing the weight between current face and one of its neighbour, existing methods only consider properties of the two faces, such as positions and normals.
%However if the two faces belong to different regions which are separated by a narrow feature edge,
%even when the two properties are close, it is improper to assign a large weight which will damage the feature.

%In this paper, we present an intrinsic mesh normal filter.
%It estimates the weight between the current face and its neighbour based on the integral of two kinds of normal differences along the geodesic path connecting their barycenters.
%Therefore, features are better preserved when removing noises.
%It is also a very generalized model and many classic filters are just special cases of it.
%Furthermore, by projection on the tangent plane, we introduce a simple pattern to estimate the weight which are faster and more effective than using geodesic path.
%We apply the filter to mesh denoising and experiments illustrate the efficacy of our method comparing with other state-of-the-art methods.

Weighted average is one of the most common strategy used in various digital signal filters, and their performances depend on the weight design.
When computing the weight between the current sample and one of its neighbours, existing methods consider only properties of the two samples, such as positions and values.
However, if the two samples belong to different regions which are separated by a narrow feature edge, even when their properties are close,
it is improper to assign a large weight which will damage the feature.

In this paper, we present an intrinsic filter on 2D manifold.
It estimates the weight between the current point and its neighbour based on the integral of some properties along the geodesic path connecting them.
Therefore, features are better preserved when removing noises.
It is also a very generalized model and many classic filters are just special cases of it.
Furthermore, by projections on the tangent plane, we introduce a simple pattern to estimate the weight, which is faster and more effective than using geodesic path.
Finally, the proposed filter is applied to mesh denoising, and experiments illustrate the efficacy of our method comparing with state-of-the-art methods.

\begin{classification} % according to http://www.acm.org/about/class/1998
\CCScat{Computer Graphics}{I.3.3}{Picture/Image Generation}{Line and curve generation}
\end{classification}

\end{abstract}