%\section{Bilateral Normal Filtering}
%Their method considers normals as a surface signal defined over the original mesh.
%This allows the design of a novel bilateral normal filter that depends on both spatial distance and signal distance.
%Their bilateral normal filtering not only was used in a local, iterative scheme, but also in a global non-iterative scheme for isotropic denoising.
%
%extremely noise,  feature preserving, irregular surface sampling.
\section{Propagation Mesh Filtering}

%�Ѻ���� * ȥ���ǵ���
\begin{figure*}
\centering
\includegraphics[width = 12.0cm]{smileface.jpg}
\vspace{0.5mm}
\caption{ The pattern}
\label{pattern}
\end{figure*}

%\begin{figure*}[!t]
%\centering
%\subfigure[Number of elements read] {\includegraphics[height=2in,width=2in,angle=-90]{52.eps}}
%\subfigure[Size of disk files scanned] {\includegraphics[height=2in,width=2in,angle=-90]{51.eps}}
%\subfigure[Execution time] {\includegraphics[height=2in,width=2in,angle=-90]{53.eps}}
%\caption{ PathStack versus TJFast using XMark data }
%\label{fig5}
%\end{figure*}

In this section, we introduce our propagated filtering framework in detail.
First, we will review the classical signal filtering theory.

\subsection{Signal filtering theory}
Given a noisy signal $I$, the filtered output $J_{p}$ at point $p$ can be produced by this general form
\begin{equation}
\begin{aligned}
\label{generalform}
J_{p} = \frac{1}{W_{p}}\int_{\mathcal{N}(p)}\omega_{p, q}I_{q}d_{q}\, ,
\end{aligned}
\end{equation}
Here, $I_{q}$ denotes the value of the signal at point $q$; $\mathcal{N}(p)$ is the neighborhood of the point $p$;
$\omega_{p, q}$ indicates the weight for each neighboring signal,
and $W_{p} = \int_{\mathcal{N}(p)}\omega_{p, q}d_{q}$ is the normalization factor for ensuring the sum of all $\omega_{p,q}$ equal to 1.
The construction of $\omega_{p,q}$ is very important, it directly relates to the performance of edge-preserving.

The classical nonlinear bilateral filtering uses two Gaussian functions as the spatial and range weights respectively:
\begin{equation}
\begin{aligned}
\label{eq:generalform}
\omega_{p,q} = g(d(p, q); \sigma_{s})g(d(I_{p}, I_{q}); \sigma_{r})\, ,
\end{aligned}
\end{equation}
where $\sigma_{s}$, $\sigma_{r}$ are variance parameters and the function $d$ defined as:
\begin{equation}
\begin{aligned}
\label{eq:generalform}
d(x_{1}, x_{2}) = ||x_{1} - x_{2} ||^{2}\, ,
%\\
%d_{BF}(I_{p}, I_{q}) = || I_{p} - I_{q} ||^{2}\, ,
\end{aligned}
\end{equation}
 As the ability of Gaussian function, the filtered signal $J_{p}$ is influenced by the neighborhood only if their spatial and range distances are enough closed.
 As a result, the bilateral filter is able to smooth the signal while preserve the edge structures.

 The effectiveness of bilateral filter relies on the range function which reflects the local information of signal, and then applies the averaging weights as the filtered output.
 However, the difference between the input signal values $I_{p}$ and $I_{q}$ can not provides reliable prediction for that between the desired signal values $J_{p}$ and $J_{q}$.

 In order to solve this problem, researchers apply the guidance signal $G$ to provide more reliable information about the structure of the desired output.
 The famous joint bilateral filter adapts following form as the weight:
 \begin{equation}
 \begin{aligned}
 \label{eq:generalform}
 \omega_{p,q} = g(d(p, q); \sigma_{s})g(d(G_{p}, G_{q}); \sigma_{r})\, ,
 \end{aligned}
 \end{equation}
( how to obtain the guidance signal??)

Other researchers adapt cumulative difference of input signal to depict the local structure of desired signal.
Propagated filter is their representative.
It considers the cumulative difference not only between adjacent signals but also start-end signals.
Their weight is
 \begin{equation}
 \begin{aligned}
 \label{propagation}
 \omega_{p,q} = g(d^{a}(I_{p}, I_{q}); \sigma_{s})g(d^{r}(I_{p}, I_{q}); \sigma_{r})\, ,
 \end{aligned}
 \end{equation}
 here,
 \begin{equation}
 \begin{aligned}
 \label{eq:path}
 d^{a}(I_{p}, I_{q}) = \sqrt{\sum_{x, x+1\in{\phi}}||I_{x+1}-I_{x}||^{2}}\, ,
 \\
 d^{r}(I_{p}, I_{q}) = \sqrt{\sum_{x\in{\phi}}||I_{x}-I_{p}||^{2}}\, ,
 %\sqrt{\sum\limits_{x,x+1\in\phi}||I_{x+1}-I_{x}}\, ,
 \end{aligned}
 \end{equation}
Note that $\phi$ is the path connecting points $p$ and $q$.
The propagated filter inherits the advantage of bilateral filter and become more adaptive to context information of signal.

When we consider normals as a surface signal defined over the original mesh, it is easy to put the propagation filtering algorithm to mesh denoising.
%The propagated filter for a normal signal which is defined over the mesh can be written as

Given points $p$ and $q$ in a mesh, the difficulty is that we need to first determine a path connecting $p$ and $q$ for calculating the weight.
It seems like the geodesic path algorithm obtains the largest weight $\omega_{p,q}$, but it is computationally expensive when the neighborhood is big.
In considering the number of mesh faces, it will use more time.

Instead, like the paper\cite{Chang2015propagated}, we consider a particular path for filtering all faces in a mesh.
It is very difficult to directly find a fixed pattern on the mesh face.
And considering in plane is much easier.
Through local projection, path would be chosen in a particular 2D pattern which is shown in Figure~\ref{pattern}.
In our experiments, we find that this operation is about 3 times faster than applying shortest path in obtaining similar results.

\subsection{Filtering mesh geometry}
Given a noisy triangle mesh, our goal is to filter the noise while less change the mesh structure.
In order to achieve this purpose, we adapt a two-stage process.
Firstly, noisy face normals are filtered iteratively by equation~\ref{generalform} with the weight chosen by equation~\ref{propagation}.
Secondly, according to the filtered face normals, vertex positions are updated iteratively though gradient descent.

{\bfseries Filtering face normals.} For a triangle face $f_{i}$, its outward normal $n_{i}$ can be calculated by outer product easily.
Then we consider $n_{i}$ as a signal associated with the face centroid $c_{i}$.
In order to filter the face normals $n_{i}$, we need to find a path $\phi$ connecting $n_{i}$ and its neighborhood $n_{j}$.
Then a filtered face normal $\bar{n_{i}}$ is computed from the propagated mesh filtering
 \begin{equation}
 \begin{aligned}
 \label{propagatedMeshFiltering}
 %\mathclap include sum �±�
 \bar{n_{i}} = \frac{1}{W_{i}}\sum_{\mathclap{f_{j}\in\mathcal{N}_{i}}}A_{j}g(d^{a}(n_{i}, n_{j}); \sigma_{s})g(d^{r}(n_{i}, n_{j}); \sigma_{r})n_{j}\, ,
 \end{aligned}
 \end{equation}
  here,
 \begin{equation}
 \begin{aligned}
 \label{eq:path}
 d^{a}(n_{i}, n_{j}) = \sqrt{\sum_{x, x+1\in{\phi}}||n_{x+1}-n_{x}||^{2}}\, ,
 \\
 d^{r}(n_{i}, n_{j}) = \sqrt{\sum_{x\in{\phi}}||n_{x}-n_{i}||^{2}}\, ,
 %\sqrt{\sum\limits_{x,x+1\in\phi}||I_{x+1}-I_{x}}\, ,
 \end{aligned}
 \end{equation}
 where $W_{i}$ is the normalization factor,
 calculated by $||\sum\limits_{\mathclap{f_{j}\in\mathcal{N}_{i}}}A_{j}g(d^{a}(n_{i}, n_{j}); \sigma_{s})g(d^{r}(n_{i}, n_{j}); \sigma_{r})n_{j}||$ which ensures that $\bar{n_{i}}$ is a unit normal;
 $\mathcal{N}_{i}$ is the neighborhood of $f_{i}$. The set $\mathcal{N}_{i}$ consists of the $f_{i}$ and its neighborhood.
 In our paper, we use the geometrical neighborhood, defined at the paper~\cite{Zhang2015Filter}.

 {\bfseries Updating vertices.} After all face normals is filtered, the vertex positions need to be updated to cater to these new normals.
 We adopt the iterative scheme in the paper~\cite{sun2007fast} to update the vertex positions.
 Namely, for a face $f_{i}$, we calculate its updated vertex positions via the following iteration form
 \begin{equation}
 \begin{aligned}
 \label{vertexupdate}
 %\mathclap include sum �±�
 \bar{v_{i}}^{(t+1)} = \bar{v_{i}}^{(t)} + \frac{1}|{\mathcal{F}_{i}}|\sum_{j\in\mathcal{F}_{i}}\bar{n}_{j}[\bar{n}_{j}\cdot(\bar{c_{j}}^{(t)}-\bar{v_{j}}^{(t)})]\, ,
 \end{aligned}
 \end{equation}
 where $\bar{v_{j}}^{(t)}$ is the value of $\bar{v_{i}}$ in the $t-th$ iteration,
 $\mathcal{F}_{i}$ is the index set of the incident faces for $\bar{v_{i}}$,
 $|\cdot|$ denotes the cardinality of a set,
 and $\bar{c_{j}}^{(t)} = \sum_{i=1}^{3}\bar{v_{j_{i}}}^{(t)}/3$ is the centroid of the triangle $f_{j}$.
 This scheme is actually based on the orthogonality between the normal and the three edges of each face on the mesh.
 Then adopts a gradient descend process for solving that orthogonality equation in the conditions of  $\mathcal{l}_2$ error.
 In our experiments, 10 to 20 iterations are sufficient for achieving satisfactory results and approximately up to above conditions.
 Our filtered process is summarized in Algorithm~\ref{algorithm:1}.
 In the next section, we introduce a simple but effective method for selecting a particular pattern for our propagated mesh filtering.  

 \section{Path chosen for mesh filtering}
 In image, many particular patterns can be chosen for generating the path needed by propagation filtering.
 These patterns are simple and easy to be thought about.
 But in triangle mesh, it is very difficult to obtain a pattern for choosing pathes.
 The easiest way to be thought of is that applying the shortest path algorithm.
 However, this method is time consuming.
 We deal with this problem in this paper within the context of mesh denoising (and smoothing).
 
 Our method is based on the following theory: the areal coordinates of triangle segments the plane into seven regions. 
 And also it reflects the local feature of mesh. We will lately explain it.
 Areal coordinates are extremely useful in engineering applications involving triangular subdomains. 
 These make analytic integrals often easier to evaluate, and Gaussian quadrature tables are often presented in terms of area coordinates.
 In the context of a triangle, the areal coordinates of a point $P$ are equivalent to the ratios of the areas of $PBC$, $PCA$ and $PAB$ to the area of the reference triangle $ABC$, 
 shown in the figure~\ref{??}.
 Because the ratios are a plus or a minus, it divide the plane where the triangle is to seven parts.
 That gives us an approach to obtain a particular pattern solving the problem of paths.
 
 Thus for each triangle $f_i$, we can projective its neighborhood $\mathcal{N}_i$ to the plane where $f_i$ is.
 In more detail, we only projective their centroid to that plane.
 Then based on the three points of $f_i$, we divide these centroid to seven parts. 
 In this way, we give each neighbor a label that belong to a specified region. 
 And in some extend, these regions can depict the local structure of a mesh.
 After that, we sort these neighbors belong to the same region basing their normal difference to $f_i$.
 Finally, we provide a particular pattern to assign the path for matching the ordered neighbors.
 The entire process is depicted in the figure~\ref{pattern}.
 
 (explain the generation of path)
 In the next section, we will introduce the experimental results basing on our method.
 
 


 
 











